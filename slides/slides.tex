% Options for packages loaded elsewhere
% Options for packages loaded elsewhere
\PassOptionsToPackage{unicode}{hyperref}
\PassOptionsToPackage{hyphens}{url}
%
\documentclass[
  ignorenonframetext,
  aspectratio=169,
]{beamer}
\newif\ifbibliography
\usepackage{pgfpages}
\setbeamertemplate{caption}[numbered]
\setbeamertemplate{caption label separator}{: }
\setbeamercolor{caption name}{fg=normal text.fg}
\beamertemplatenavigationsymbolshorizontal
% remove section numbering
\setbeamertemplate{part page}{
  \centering
  \begin{beamercolorbox}[sep=16pt,center]{part title}
    \usebeamerfont{part title}\insertpart\par
  \end{beamercolorbox}
}
\setbeamertemplate{section page}{
  \centering
  \begin{beamercolorbox}[sep=12pt,center]{section title}
    \usebeamerfont{section title}\insertsection\par
  \end{beamercolorbox}
}
\setbeamertemplate{subsection page}{
  \centering
  \begin{beamercolorbox}[sep=8pt,center]{subsection title}
    \usebeamerfont{subsection title}\insertsubsection\par
  \end{beamercolorbox}
}
% Prevent slide breaks in the middle of a paragraph
\widowpenalties 1 10000
\raggedbottom
\usepackage{iftex}
\ifPDFTeX
  \usepackage[T1]{fontenc}
  \usepackage[utf8]{inputenc}
  \usepackage{textcomp} % provide euro and other symbols
\else % if luatex or xetex
  \usepackage{unicode-math} % this also loads fontspec
  \defaultfontfeatures{Scale=MatchLowercase}
  \defaultfontfeatures[\rmfamily]{Ligatures=TeX,Scale=1}
\fi
\usepackage{lmodern}

\usetheme[]{Madrid}
\usecolortheme[]{beaver}
\ifPDFTeX\else
  % xetex/luatex font selection
\fi
% Use upquote if available, for straight quotes in verbatim environments
\IfFileExists{upquote.sty}{\usepackage{upquote}}{}
\IfFileExists{microtype.sty}{% use microtype if available
  \usepackage[]{microtype}
  \UseMicrotypeSet[protrusion]{basicmath} % disable protrusion for tt fonts
}{}
\makeatletter
\@ifundefined{KOMAClassName}{% if non-KOMA class
  \IfFileExists{parskip.sty}{%
    \usepackage{parskip}
  }{% else
    \setlength{\parindent}{0pt}
    \setlength{\parskip}{6pt plus 2pt minus 1pt}}
}{% if KOMA class
  \KOMAoptions{parskip=half}}
\makeatother


\usepackage{longtable,booktabs,array}
\usepackage{calc} % for calculating minipage widths
\usepackage{caption}
% Make caption package work with longtable
\makeatletter
\def\fnum@table{\tablename~\thetable}
\makeatother
\usepackage{graphicx}
\makeatletter
\newsavebox\pandoc@box
\newcommand*\pandocbounded[1]{% scales image to fit in text height/width
  \sbox\pandoc@box{#1}%
  \Gscale@div\@tempa{\textheight}{\dimexpr\ht\pandoc@box+\dp\pandoc@box\relax}%
  \Gscale@div\@tempb{\linewidth}{\wd\pandoc@box}%
  \ifdim\@tempb\p@<\@tempa\p@\let\@tempa\@tempb\fi% select the smaller of both
  \ifdim\@tempa\p@<\p@\scalebox{\@tempa}{\usebox\pandoc@box}%
  \else\usebox{\pandoc@box}%
  \fi%
}
% Set default figure placement to htbp
\def\fps@figure{htbp}
\makeatother





\setlength{\emergencystretch}{3em} % prevent overfull lines

\providecommand{\tightlist}{%
  \setlength{\itemsep}{0pt}\setlength{\parskip}{0pt}}



 


\makeatletter
\@ifpackageloaded{caption}{}{\usepackage{caption}}
\AtBeginDocument{%
\ifdefined\contentsname
  \renewcommand*\contentsname{Table of contents}
\else
  \newcommand\contentsname{Table of contents}
\fi
\ifdefined\listfigurename
  \renewcommand*\listfigurename{List of Figures}
\else
  \newcommand\listfigurename{List of Figures}
\fi
\ifdefined\listtablename
  \renewcommand*\listtablename{List of Tables}
\else
  \newcommand\listtablename{List of Tables}
\fi
\ifdefined\figurename
  \renewcommand*\figurename{Figure}
\else
  \newcommand\figurename{Figure}
\fi
\ifdefined\tablename
  \renewcommand*\tablename{Table}
\else
  \newcommand\tablename{Table}
\fi
}
\@ifpackageloaded{float}{}{\usepackage{float}}
\floatstyle{ruled}
\@ifundefined{c@chapter}{\newfloat{codelisting}{h}{lop}}{\newfloat{codelisting}{h}{lop}[chapter]}
\floatname{codelisting}{Listing}
\newcommand*\listoflistings{\listof{codelisting}{List of Listings}}
\makeatother
\makeatletter
\makeatother
\makeatletter
\@ifpackageloaded{caption}{}{\usepackage{caption}}
\@ifpackageloaded{subcaption}{}{\usepackage{subcaption}}
\makeatother

\usepackage{bookmark}
\IfFileExists{xurl.sty}{\usepackage{xurl}}{} % add URL line breaks if available
\urlstyle{same}
\hypersetup{
  pdftitle={COSIMO-IDAI Generative AI Series: Agentic Coding},
  pdfauthor={Fabian Spill},
  hidelinks,
  pdfcreator={LaTeX via pandoc}}


\title{COSIMO-IDAI Generative AI Series: Agentic Coding}
\subtitle{From Idea to Proof-of-Concept in One Hour}
\author{Fabian Spill}
\date{2026-02-13}
\institute{School of Mathematics, University of Birmingham}

\begin{document}
\frame{\titlepage}


\begin{frame}{Today's Agenda}
\phantomsection\label{todays-agenda}
\begin{itemize}[<+->]
\tightlist
\item
  \textbf{Brief overview} of the aim of this series
\item
  \textbf{Brief overview} of coding agents
\item
  \textbf{Live demonstration} of addressing a \emph{scientific} problem
  using AI
\item
  \textbf{Example domains}: SIR modelling and simulation code, data
  analysis pipeline
\end{itemize}

\note{Welcome everyone. Today we'll see how AI coding agents can
accelerate research.}
\end{frame}

\begin{frame}{COSIMO-IDAI Generative AI Series: Accelerating Science
Through Appropriate Use of AI}
\phantomsection\label{cosimo-idai-generative-ai-series-accelerating-science-through-appropriate-use-of-ai}
\textbf{Overall Purpose:} AI is now integral to science, but we must
optimize for \textbf{both speed and robustness}.

\begin{columns}[T]
\begin{column}{0.5\linewidth}
\textbf{The Goal: Shift the Pareto Front} We don't just want to move
\emph{faster} along the same trade-off curve.

We want to \textbf{expand the frontier}, enabling research that is both
rapid \textbf{and} reproducible.
\end{column}

\begin{column}{0.5\linewidth}
\pandocbounded{\includegraphics[keepaspectratio]{assets/pareto_front.png}}
\end{column}
\end{columns}

\textbf{This series aims to:}

\begin{itemize}[<+->]
\tightlist
\item
  Equip researchers with \textbf{practical, model-agnostic skills}
\item
  Develop \textbf{critical literacy} on failure modes
\item
  Promote \textbf{best practices} for reproducibility \& validation
\item
  Create a shared \textbf{institutional culture} around AI science
\end{itemize}

\note{Target audience: PhD students, postdocs, PIs, and RSEs across all
disciplines.}
\end{frame}

\begin{frame}{COSIMO-IDAI Generative AI Series}
\phantomsection\label{cosimo-idai-generative-ai-series}
\begin{itemize}[<+->]
\tightlist
\item
  Today's Topic: From Idea to Proof-of-Concept in One Hour, Fabian Spill
\item
  27/2/2026: Multi-agentic Workflows: Orchestrating Coding Agents with
  CLI Tools, skills.md, and MCPs, Cesar Contreras
\item
  TBC: Ethical Considerations of Generative AI in Research, Kamilla
  Kopec-Harding
\item
  TBC: Generative AI in Education, Generative AI for Grant Writing
\item
  TBC: You
\end{itemize}

\note{}
\end{frame}

\begin{frame}{Join the Community!}
\phantomsection\label{join-the-community}
\textbf{We are building a Community of Practice.}

\begin{itemize}[<+->]
\tightlist
\item
  📢 \textbf{Contribute}: Share your prompts or workflows in future
  sessions
\item
  🏳️ \textbf{Skeptics Welcome}: Present AI failures --- are they
  inherent or fixable?
\end{itemize}
\end{frame}

\begin{frame}{Disclaimer: Important Policy Notice}
\phantomsection\label{disclaimer-important-policy-notice}
\textbf{University Policy (as of Feb 2026):} Only \textbf{M365 Copilot}
is officially supported. Tests for compliant advanced models are
underway.

\textbf{This session is for information only.}

❌ \textbf{DO NOT} use these tools with: - Sensitive / Personal Data
(GDPR) - Export Control / Restricted IP - Clinical Data

💡 \textbf{Best Practice:} Use \textbf{isolated environments} (VMs,
containers, clean accounts), control permissions and data provenance.
\end{frame}

\begin{frame}[fragile]{Part 1: A Brief History of AI Coding Assistants:
From Autocomplete to Agents}
\phantomsection\label{part-1-a-brief-history-of-ai-coding-assistants-from-autocomplete-to-agents}
\textbf{Two Converging Paths:}

\begin{enumerate}
\tightlist
\item
  \textbf{From Autocomplete:} \texttt{Tab-Complete} → Copilot → ``Ghost
  text''
\item
  \textbf{From Chatbots:} ChatGPT → Copy/Paste → Integrated Chat
\item
  \textbf{The Merger:} Coding Agents
\end{enumerate}
\end{frame}

\begin{frame}{Path 1: Autocomplete (The ``Inline'' Path)}
\phantomsection\label{path-1-autocomplete-the-inline-path}
\begin{itemize}
\tightlist
\item
  \textbf{Start:} IDE suggestions (variable names)
\item
  \textbf{Evolution:} GitHub Copilot (2021) --- context-aware next
  tokens
\item
  \textbf{Limitation:}

  \begin{itemize}
  \tightlist
  \item
    Reactive, not proactive
  \item
    Cannot run or test its own code
  \item
    ``Thinks'' in small context windows (file-level)
  \end{itemize}
\end{itemize}
\end{frame}

\begin{frame}{Path 2: Chatbots (The ``Conversational'' Path)}
\phantomsection\label{path-2-chatbots-the-conversational-path}
\begin{itemize}
\tightlist
\item
  \textbf{Interaction:} Copy-paste into ChatGPT/Claude window
\item
  \textbf{Limitation:}

  \begin{itemize}
  \tightlist
  \item
    \textbf{The ``Clipboard Wall'':} You must manually move code back
    and forth
  \item
    \textbf{Zero-shot:} If it fails, \emph{you} paste the error back
  \item
    \textbf{Context:} Limited to what you paste
  \end{itemize}
\end{itemize}
\end{frame}

\begin{frame}[fragile]{The Convergence: Coding Agents}
\phantomsection\label{the-convergence-coding-agents}
\begin{itemize}
\tightlist
\item
  \textbf{Interaction:} Agent has \textbf{terminal}/\textbf{ide} and
  \textbf{file access}
\item
  \textbf{The Paradigm Shift:}

  \begin{itemize}
  \tightlist
  \item
    Old: REPL (Read-Eval-Print) controlled by \textbf{Human}
  \item
    New: \textbf{Plan-Act-Observe} controlled by \textbf{Agent}
  \end{itemize}
\item
  \textbf{The Loop:}

  \begin{enumerate}
  \tightlist
  \item
    \textbf{Plan:} ``Calculate pairwise distances between atoms''
  \item
    \textbf{Act:} Writes script using appropriate libraries
  \item
    \textbf{Observe:} Sees \texttt{MemoryError}
  \item
    \textbf{Debugs and Refines:} ``Switch to sparse matrix''
  \end{enumerate}
\item
  \textbf{Why it matters:} Closes the feedback loop \emph{without human
  intervention}
\end{itemize}
\end{frame}

\begin{frame}{The AI Landscape (2026): Models \& Tools}
\phantomsection\label{the-ai-landscape-2026-models-tools}
\begin{block}{I. The Brains (The Intelligence)}
\phantomsection\label{i.-the-brains-the-intelligence}
\begin{longtable}[]{@{}
  >{\raggedright\arraybackslash}p{(\linewidth - 4\tabcolsep) * \real{0.1714}}
  >{\raggedright\arraybackslash}p{(\linewidth - 4\tabcolsep) * \real{0.2857}}
  >{\raggedright\arraybackslash}p{(\linewidth - 4\tabcolsep) * \real{0.5429}}@{}}
\toprule\noalign{}
\begin{minipage}[b]{\linewidth}\raggedright
Tier
\end{minipage} & \begin{minipage}[b]{\linewidth}\raggedright
Use Case
\end{minipage} & \begin{minipage}[b]{\linewidth}\raggedright
Key Models (2026)
\end{minipage} \\
\midrule\noalign{}
\endhead
\textbf{Fast / General} & Autocomplete, simple refactors, speed &
GPT-4o-2026, Claude 3.5 Haiku, Gemini 2.0 Flash \\
\textbf{Thinking} & Hard logic, architecture, debugging & GPT-5.2 Think,
DeepSeek R2, Claude 4.5 Sonnet \\
\textbf{Pro / Frontier} & Complex maths, massive context & GPT-5.2 Pro,
Gemini DeepThink, Claude 4.6 Opus \\
\textbf{Open Weights} & Local / Private / Cheap & Llama 4, Mistral
Large, DeepSeek V3, Kimi K2.5 \\
\bottomrule\noalign{}
\end{longtable}
\end{block}

\begin{block}{II. The Hands (The Agentic Tools)}
\phantomsection\label{ii.-the-hands-the-agentic-tools}
\begin{longtable}[]{@{}
  >{\raggedright\arraybackslash}p{(\linewidth - 4\tabcolsep) * \real{0.2778}}
  >{\raggedright\arraybackslash}p{(\linewidth - 4\tabcolsep) * \real{0.4444}}
  >{\raggedright\arraybackslash}p{(\linewidth - 4\tabcolsep) * \real{0.2778}}@{}}
\toprule\noalign{}
\begin{minipage}[b]{\linewidth}\raggedright
Category
\end{minipage} & \begin{minipage}[b]{\linewidth}\raggedright
Interface Mode
\end{minipage} & \begin{minipage}[b]{\linewidth}\raggedright
Examples
\end{minipage} \\
\midrule\noalign{}
\endhead
\textbf{IDE integrated} & ``Co-pilot'' / Chat Sidebar & Cursor, GitHub
Copilot \\
\textbf{Autonomous / CLI} & ``Go do this task'' & Cline, Claude Code,
Gemini CLI, Codex CLI \\
\textbf{Multi-Agent / Managers} & Architect / Orchestrator & Codex
Desktop, Antigravity \\
\bottomrule\noalign{}
\end{longtable}
\end{block}
\end{frame}

\begin{frame}{Key Insight: Be Specific about the \emph{Goal}, not the
\emph{Syntax}}
\phantomsection\label{key-insight-be-specific-about-the-goal-not-the-syntax}
❌ Write a for loop that iterates through the list and calculates the
mean

✅ Implement an SIR epidemic model with transmission rate β and recovery
rate γ

The agent knows how to write loops. \textbf{You} know what an SIR model
is.

\textbf{Rule of Thumb:} - \textbf{Don't} micromanage the code (syntax) -
\textbf{Do} micromanage the output (science/style)
\end{frame}

\begin{frame}{Tools We'll Use Today}
\phantomsection\label{tools-well-use-today}
\begin{longtable}[]{@{}
  >{\raggedright\arraybackslash}p{(\linewidth - 2\tabcolsep) * \real{0.4000}}
  >{\raggedright\arraybackslash}p{(\linewidth - 2\tabcolsep) * \real{0.6000}}@{}}
\toprule\noalign{}
\begin{minipage}[b]{\linewidth}\raggedright
Tool
\end{minipage} & \begin{minipage}[b]{\linewidth}\raggedright
Purpose
\end{minipage} \\
\midrule\noalign{}
\endhead
\textbf{Agent-enabled IDE} & VS Code + agent integration (Cursor /
Copilot / Codex Desktop / etc.) \\
\textbf{LLM backend} & OpenAI / Anthropic / Google / open-weight local
models \\
\textbf{Python + SciPy stack} & numpy, scipy, pandas, matplotlib \\
\bottomrule\noalign{}
\end{longtable}

\note{These tools are accessible to anyone with basic coding
familiarity.}
\end{frame}

\begin{frame}{Part 2: From Blank File to Prototype}
\phantomsection\label{part-2-from-blank-file-to-prototype}
We'll build a simple epidemiological simulation \textbf{from scratch}.

\textbf{Live Demo Ahead!} 🎬
\end{frame}

\begin{frame}{The Challenge: Our Goal Today}
\phantomsection\label{the-challenge-our-goal-today}
\textbf{Remember the pandemic?} The famous \textbf{R number}?

We want to simulate that spread.

\[ \text{Susceptible} \rightarrow \text{Infected} \rightarrow \text{Recovered} \]

\textbf{The Mission:}

\begin{enumerate}
\tightlist
\item
  \textbf{Build} this simulation code from scratch.
\item
  \textbf{Visualise} the outbreak.
\item
  \textbf{Analyse} it with data.
\end{enumerate}

\emph{\ldots without writing the code ourselves.}
\end{frame}

\begin{frame}{The Math Behind the Model}
\phantomsection\label{the-math-behind-the-model}
\begin{columns}[T]
\begin{column}{0.6\linewidth}
We want to go from \textbf{blank file} to \textbf{working simulation} of
a pandemic (SIR model).

\begin{itemize}
\tightlist
\item
  \textbf{S}: Susceptible
\item
  \textbf{I}: Infected
\item
  \textbf{R}: Recovered
\item
  \textbf{N}: Total population (constant)
\end{itemize}

We will build this \emph{live} to demonstrate the power of
\textbf{Coding Agents}.
\end{column}

\begin{column}{0.4\linewidth}
\[
\begin{aligned}
\frac{dS}{dt} &= -\beta \frac{S I}{N} \\
\frac{dI}{dt} &= \beta \frac{S I}{N} - \gamma I \\
\frac{dR}{dt} &= \gamma I
\end{aligned}
\]
\end{column}

\textbf{The ``R Number'' (\(R_0\)):}

\[ R_0 = \frac{\beta}{\gamma} \]

\begin{itemize}
\tightlist
\item
  If \(R_0 > 1\): Epidemic grows 📈
\item
  If \(R_0 < 1\): Epidemic dies out 📉
\end{itemize}
\end{columns}
\end{frame}

\begin{frame}{Prompt 1: Project Initialization}
\phantomsection\label{prompt-1-project-initialization}
Create a Python project structure for simulating an SIR epidemic model.
Include a main simulation script, a module for the model equations, and
a visualization script. Set out the structure without implementing any
details.

Let's see what the agent produces\ldots{}
\end{frame}

\begin{frame}{Prompt 2: Implement the Model}
\phantomsection\label{prompt-2-implement-the-model}
Implement the SIR model differential equations using the standard
population-normalized form (β\emph{S}I/N). The model should track
Susceptible, Infected, and Recovered populations over time. Use
scipy.integrate.odeint for integration.
\end{frame}

\begin{frame}{Prompt 3: Visualization}
\phantomsection\label{prompt-3-visualization}
Create a visualization that shows S, I, R curves over time. Add proper
labels, legend, and a title. Use a clean, publication-ready style.
\end{frame}

\begin{frame}{Prompt 4: Explain \& Review}
\phantomsection\label{prompt-4-explain-review}
Look at the plot. Explain what is happening. What is the peak infected
count and when does it occur?

\emph{Key Insight:} The agent acts as a \textbf{tutor} and
\textbf{reviewer}.
\end{frame}

\begin{frame}{Talking to AI with Scientific Intent}
\phantomsection\label{talking-to-ai-with-scientific-intent}
\textbf{Effective prompting strategies:}

\begin{enumerate}[<+->]
\tightlist
\item
  \textbf{State the scientific goal} first, not the implementation
\item
  \textbf{Use domain terminology} --- the model understands SIR, ODE,
  etc.
\item
  \textbf{Provide constraints} --- ``use scipy'' or ``make it efficient
  for large N''
\item
  \textbf{Iterate} --- refine based on initial output
\end{enumerate}
\end{frame}

\begin{frame}{Part 3: Fitting Real Data}
\phantomsection\label{part-3-fitting-real-data}
\textbf{Live Demo!} 🎬\\
We'll fit our SIR model to real pandemic data from Italy's first wave.
\end{frame}

\begin{frame}{The Data: Italy, February--May 2020}
\phantomsection\label{the-data-italy-februarymay-2020}
\begin{itemize}
\tightlist
\item
  \textbf{Source:} Johns Hopkins CSSE COVID-19 Repository
\item
  \textbf{Period:} First wave (before vaccines, clear lockdown signal)
\item
  \textbf{Key event:} National lockdown on \textbf{March 9, 2020}
\end{itemize}

\emph{Can our simple SIR model capture a real pandemic?}
\end{frame}

\begin{frame}[fragile]{Prompt 1: Load Real Data}
\phantomsection\label{prompt-1-load-real-data}
Load the COVID-19 data for Italy from
\texttt{data/covid\_italy\_first\_wave.csv}. Plot active infections over
time. Add a vertical line for the lockdown date (March 9).
\end{frame}

\begin{frame}[fragile]{Prompt 2: Fit the SIR Model}
\phantomsection\label{prompt-2-fit-the-sir-model}
Create a function that fits our SIR model to the Italy data. Use
\texttt{scipy.optimize.minimize} to find the best β and γ. Print the
estimated R0.
\end{frame}

\begin{frame}{Prompt 3: Visualize and Critique}
\phantomsection\label{prompt-3-visualize-and-critique}
Plot the best-fit model against the real data. Critically evaluate:
where does the simple SIR assumption break down?

\emph{Key insight:} The lockdown changed β --- our constant-β model
can't capture that!
\end{frame}

\begin{frame}{Interpret the Fitted Parameters}
\phantomsection\label{interpret-the-fitted-parameters}
Based on the optimized β and γ: What is R₀? How does it compare to
published estimates (2.5-3.5)? What is the implied infection duration
(1/γ)?

\emph{Teaching point:} R₀ depends on context; not an intrinsic property
of the virus alone.
\end{frame}

\begin{frame}{Part 4: Going Beyond Basic SIR}
\phantomsection\label{part-4-going-beyond-basic-sir}
\begin{itemize}[<+->]
\tightlist
\item
  \textbf{4.1 Time-Varying β:} Model the lockdown effect with piecewise
  transmission
\item
  \textbf{4.2 Country Comparison:} Compare Italy vs South Korea response
  strategies
\item
  \textbf{4.3 Uncertainty:} Bootstrap confidence intervals for R₀
\end{itemize}
\end{frame}

\begin{frame}{Prompt: Time-Varying Transmission}
\phantomsection\label{prompt-time-varying-transmission}
Modify the SIR model to use β\_pre before lockdown and β\_post after.
Fit both values and compare R₀ before and after March 9.

\emph{This captures what our constant-β model missed!}
\end{frame}

\begin{frame}{Prompt: Country Comparison}
\phantomsection\label{prompt-country-comparison}
Repeat the SIR fitting for South Korea. Create a side-by-side comparison
with Italy. Which country had lower R₀? What strategies might explain
this?

\emph{South Korea used aggressive testing vs Italy's lockdown approach.}
\end{frame}

\begin{frame}{Prompt: Uncertainty Quantification}
\phantomsection\label{prompt-uncertainty-quantification}
Use bootstrap resampling to estimate 95\% confidence intervals for R₀.
Visualize the uncertainty as a shaded band around the model curve.

\textbf{Point estimates without uncertainty can be misleading!}
\end{frame}

\begin{frame}{Part 5: Let the AI Write the Report}
\phantomsection\label{part-5-let-the-ai-write-the-report}
``Write a scientific report summarizing our COVID-19 SIR analysis.
Include Introduction, Methods, Results with figures, Discussion of
limitations, and Conclusion.''
\end{frame}

\begin{frame}{Prompt: Peer Review}
\phantomsection\label{prompt-peer-review}
``Act as a peer reviewer. Evaluate the report for clarity, accuracy, and
completeness. Provide constructive feedback.''

\emph{The AI can both write AND review!}
\end{frame}

\begin{frame}{Part 6: Outlook (2026)}
\phantomsection\label{part-6-outlook-2026}
\begin{itemize}[<+->]
\tightlist
\item
  \textbf{2022}: Copy-pasting fragments
\item
  \textbf{2026}: Orchestrating \textbf{Agents}
\item
  \textbf{New Skills}: Managing \textbf{MCP Servers}, defining Agent
  Skills
\end{itemize}
\end{frame}

\begin{frame}{Coming Up Next}
\phantomsection\label{coming-up-next}
\textbf{In Two Weeks:} ``Building Custom Agent Skills'' (Connecting
agents to your own databases and tools)

\textbf{Today:} We focus on the \emph{practical application}.
\end{frame}

\begin{frame}{Wrap-Up: What We Covered}
\phantomsection\label{wrap-up-what-we-covered}
\begin{itemize}[<+->]
\tightlist
\item
  ✅ Coding agents as scientific collaborators
\item
  ✅ Prompting with \emph{scientific intent}, not syntax
\item
  ✅ Building simulation code from scratch
\item
  ✅ Fitting models to real COVID-19 data
\item
  ✅ AI-assisted scientific report writing
\end{itemize}
\end{frame}

\begin{frame}{Discussion Points}
\phantomsection\label{discussion-points}
\begin{longtable}[]{@{}
  >{\raggedright\arraybackslash}p{(\linewidth - 2\tabcolsep) * \real{0.3043}}
  >{\raggedright\arraybackslash}p{(\linewidth - 2\tabcolsep) * \real{0.6957}}@{}}
\toprule\noalign{}
\begin{minipage}[b]{\linewidth}\raggedright
Topic
\end{minipage} & \begin{minipage}[b]{\linewidth}\raggedright
Teaching Point
\end{minipage} \\
\midrule\noalign{}
\endhead
\textbf{Model-data mismatch} & SIR is too simple for COVID-19; useful
for intuition \\
\textbf{Data quality} & ``Recovered'' poorly reported; active cases are
estimates \\
\textbf{Interventions} & Constant β fails when behavior changes \\
\textbf{R₀ interpretation} & Depends on context, not intrinsic to
virus \\
\textbf{Extensions} & SEIR, age structure, spatial models \\
\bottomrule\noalign{}
\end{longtable}
\end{frame}

\begin{frame}{Key Takeaways}
\phantomsection\label{key-takeaways}
\begin{enumerate}
\tightlist
\item
  \textbf{You bring domain expertise} --- the AI brings coding fluency
\item
  \textbf{Iterate quickly} --- first draft is rarely perfect
\item
  \textbf{Trust but verify} --- always review the generated code
\item
  \textbf{Scientific problems need scientific prompts}
\end{enumerate}
\end{frame}

\begin{frame}[fragile]{Resources}
\phantomsection\label{resources}
\begin{itemize}
\tightlist
\item
  \textbf{This repo}:
  \texttt{github.com/fab321/COSIMO\_IDAI\_Agentic\_Coding\_Tutorial}
\item
  \textbf{Prompts used today}:
  \texttt{planning/contents\_brainstorming.md} §6
\item
  \textbf{Example outputs}: \texttt{examples/} directory
\end{itemize}

\note{Materials will be available after the seminar.}
\end{frame}

\begin{frame}[fragile]{Data Source Citation}
\phantomsection\label{data-source-citation}
\textbf{JHU CSSE COVID-19 Dataset}

\begin{quote}
Dong E, Du H, Gardner L. An interactive web-based dashboard to track
COVID-19 in real time. \emph{Lancet Inf Dis.} 2020;20(5):533-534.
\end{quote}

Repository: \texttt{github.com/CSSEGISandData/COVID-19}
\end{frame}

\begin{frame}{Questions?}
\phantomsection\label{questions}
🎤 \textbf{Open Discussion}

Networking lunch follows at 12:00!
\end{frame}

\begin{frame}{Thank You!}
\phantomsection\label{thank-you}
Fabian Spill\\
f.spill@bham.ac.uk
\end{frame}




\end{document}
